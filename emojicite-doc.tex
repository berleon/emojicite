\documentclass{l3doc}
\usepackage{fontspec}
\usepackage{emojicite}
\usepackage{listings}
\usepackage{float}
\usepackage{array,booktabs,fontspec,geometry,graphicx,longtable,xcolor}

\setcitestyle{authoryear, round}
\bibliographystyle{plainnat}
\hypersetup{citecolor=black}


\definecolor{mygreen}{rgb}{0,0.6,0}
\definecolor{mygray}{rgb}{0.5,0.5,0.5}
\definecolor{mymauve}{rgb}{0.58,0,0.82}

\lstset{
  backgroundcolor=\color{white},   % choose the background color; you must add \usepackage{color} or \usepackage{xcolor}; should come as last argument
  basicstyle=\small,        % the size of the fonts that are used for the code
  breakatwhitespace=false,         % sets if automatic breaks should only happen at whitespace
  breaklines=true,                 % sets automatic line breaking
  captionpos=b,                    % sets the caption-position to bottom
  commentstyle=\color{mygreen},    % comment style
  deletekeywords={...},            % if you want to delete keywords from the given language
  escapeinside={\%*}{*)},          % if you want to add LaTeX within your code
  extendedchars=true,              % lets you use non-ASCII characters; for 8-bits encodings only, does not work with UTF-8
  firstnumber=1,                % start line enumeration with line 1000
  frame=single,	                   % adds a frame around the code
  keepspaces=true,                 % keeps spaces in text, useful for keeping indentation of code (possibly needs columns=flexible)
  keywordstyle=\color{blue},       % keyword style
  language=Octave,                 % the language of the code
  morekeywords={*,...},            % if you want to add more keywords to the set
  numbers=left,                    % where to put the line-numbers; possible values are (none, left, right)
  numbersep=5pt,                   % how far the line-numbers are from the code
  numberstyle=\tiny\color{mygray}, % the style that is used for the line-numbers
  rulecolor=\color{black},         % if not set, the frame-color may be changed on line-breaks within not-black text (e.g. comments (green here))
  showspaces=false,                % show spaces everywhere adding particular underscores; it overrides 'showstringspaces'
  showstringspaces=false,          % underline spaces within strings only
  showtabs=false,                  % show tabs within strings adding particular underscores
  stepnumber=1,                    % the step between two line-numbers. If it's 1, each line will be numbered
  stringstyle=\color{mymauve},     % string literal style
  tabsize=2,	                   % sets default tabsize to 2 spaces
  title=\lstname                   % show the filename of files included with \lstinputlisting; also try caption instead of title
}

\setlength{\parskip}{0.0em}
\setlength\parindent{0pt}

\makeatletter
\ExplSyntaxOn
\def\@fnsymbol#1{
  \ensuremath{
    \ifcase #1
        \or \text{\emoji{globe-with-meridians}}
        \or \text{\emoji{email}}
    \fi}}
\ExplSyntaxOff
\makeatother


\title{The \pkg{emojicite} package \thanks{\url{https://github.com/berleon/emojicite}} \\
  Adds Emojis to Citations \\
  \normalsize (requires Lua\LaTeX)}
  \author{Leon Sixt \thanks{\url{mail@leon-sixt.com}}}
\date{\emoji{date} 2020/04/13\quad v0.1}


\begin{document}
\maketitle

\tableofcontents

\setlength{\parskip}{0.5em}

\section{Introduction}

Scientific publications are too dry. Too much math, too little emotions. Science
needs emojis!
Leave a small heart to value the hard work gone into a paper \emojicitep{smith2014honey, heart}.
Finally, you can express what you true think directly as in \emojicitep{wakefield1998retracted, facepalm}.
We could also indicate, how thoroughly we read papers \emojicitep{van2014difference, see-no-evil}.

\begin{verbatim}
    \emojicitep{van2014difference, see-no-evil}.
\end{verbatim}


The package is based on the \pkg{emoji} package\footnote{\url{https://ctan.org/pkg/emoji}}.
See their documentation for the emoji codes\footnote{\url{http://mirrors.ctan.org/macros/luatex/latex/emoji/emoji-doc.pdf}}. You can also use \texttt{\textbackslash setemojifont\{\dots\}} to select a different emoji font.


%\emojicitep{wakefield1998retracted, facepalm, see-no-evil}
You can use up to two emojis \emojicitep{shannon1948, bow, thinking}. The package also supports
to cite multiple works as in \emojicitep{sixt2019explanations, selfie; adebayo2018sanity, +1}.
\pkg{emojicite} does not support more than two emojis.
\texttt{\textbackslash emojicitep\{wakefield1998retracted, facepalm, roll-eyes, shrug\}}
renders as \emojicitep{wakefield1998retracted, facepalm, roll-eyes, shrug}.
Let's ensure science does not get to emotional \emoji{point-up-2}\emoji{neutral-face}.

\newpage
\subsection{Example}

\lstinputlisting[language=Tex,caption={Output of the example is: \emojicitep{einstein, kissing-heart} }]{example.tex}
\subsection{Requirements}

The package works with the Tex Live 2020 distributon and you need to use \texttt{lualatex}.
If you use the \texttt{latexmk} tool, use it with this flag: \texttt{latexmk -pdflua}.
See the \pkg{emoji} package for are an in-depth description of the requirements.


\section{Examplary Usage}

Here are some emoji, I found fitting.

\begin{table}[H]
    \centering
    \caption{How did you liked the cited work?}
    \begin{tabular}{l l l}
        \hline \hline
        \textbf{Citation} & \textbf{Emoji} & \textbf{Description} \\ \hline \hline
        \emojicitep{einstein, kissing-heart} & \texttt{kissing-heart} & I like this work. Here is a kiss. \\ \hline
        \emojicitep{shannon1948, bow} &\texttt{bow} & Wow, I can only bow to this work. \\ \hline
        \emojicitep{kim2017interpretability, +1} &\texttt{+1} & Good work! \\ \hline
        \emojicitep{zhang20167kissing, confused} &\texttt{confused} & I am confused by this work. \\ \hline
        \emojicitep{le1989gemini, yawning-face}  & \texttt{yawning-face} & Boring work.\\ \hline
        \emojicitep{tishby2015deep, raised-eyebrow} &\texttt{raised-eyebrow}& I have some serious questions...\\ \hline
        %\emojicitep{shannon1948, thumbs-down} &\texttt{thumbs-down} & As a reviewer, I would not accept this. \\ \hline
        \emojicitep{wakefield1998retracted, facepalm} &\texttt{facepalm}& omg, this work sucks! \\ \hline
    \end{tabular}
\end{table}

\begin{table}[H]
    \centering
    \caption{How thoroughly have you read the work?}
    \begin{tabular}{l l l}
        \hline \hline
        \textbf{Citation} & \textbf{Emoji} & \textbf{Description} \\ \hline \hline
        \emojicitep{kingma2013auto, nerd-face}  & \texttt{nerd-face}& I know everything about this work. \\ \hline
        \emojicitep{kim2017interpretability, graduation-cap}  & \texttt{graduation-cap}& I know this work well. \\ \hline
        \emojicitep{shannon1948, thinking} &\texttt{thinking}& I read it but I still have questions.  \\ \hline
        \emojicitep{jones1972statistical, see-no-evil} &\texttt{see-no-evil}& Ups, I did not read this work in-depth.  \\ \hline
        \emojicitep{einstein, shrug} &\texttt{shrug}& too long; did not read  \\ \hline
    \end{tabular}
\end{table}


\begin{table}[H]
    \centering
    \caption{Special emojis}
    \begin{tabular}{l l l}
        \hline \hline
        \textbf{Citation} & \textbf{Emoji} & \textbf{Description} \\ \hline \hline
        \emojicitep{sixt2019explanations, selfie}  & \texttt{selfie} & Citing myself. It just fits perfectly.\\ \hline
        \emojicitep{blei2003latent, writing-hand, +1}  & \texttt{writing-hand, +1} & Well written work.\\ \hline
    \end{tabular}
\end{table}

\begin{table}[H]
    \centering
    \caption{Short summary of papers}
    \begin{tabular}{l l l}
        \hline \hline
        \textbf{Citation} & \textbf{Emoji} & \textbf{Description} \\ \hline \hline
        \emojicitep{watson1953molecular, dna}  & \texttt{dna} & Helix structure of DNA.\\ \hline
        \emojicitep{smith2014honey, grimacing, bee}  & \texttt{grimacing, bee} & Getting sting by honey bees.  \\ \hline
        \emojicitep{zhang20167kissing, kissing-heart, robot}  & \texttt{kissing-heart, robot} & Kissing machine.  \\ \hline
        \emojicitep{nakanishi2014remote, handshake, mechanical-arm}  & \texttt{handshake, mechanical-arm} & Handshake machine.  \\ \hline
    \end{tabular}
\end{table}


\section{Limitations}

Currently, only the name-year format is supported but using a number format is also simple wrong.
Unfortunatly, the Tex Live package required by this verison is not yet available
on ArXiv. If you want to publish on ArXiv, you probably would have to upload the PDF.
Another problem could be that scientific journal and conferences might not see the same great value in emojis.
Every revolution requires a bit of time.



\bibliography{bibliography}
\end{document}
